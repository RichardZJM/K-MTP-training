\documentclass[9pt,twocolumn,twoside]{opticajnl}
\journal{opticajournal} % use for journal or Optica Open submissions

% See template introduction for guidance on setting shortarticle option
\setboolean{shortarticle}{false}
% true = letter/tutorial
% false = research/review article

% ONLY applicable for journal submission shortarticle types:
% When \setboolean{shortarticle}{true}
% then \setboolean{memo}{true} will print "Memorandum" on title page header
% Otherwise header will remain as "Letter"
% \setboolean{memo}{true}

\usepackage{lineno}
\linenumbers % Turn off line numbering for Optica Open preprint submissions.

\title{Training and Validation of Moment Tensor Potentials for Potassium Metal}

\author[1,2,3]{ Richard Meng}
\author[1]{Laurent  Béland}
\author[1]{Hao Sun}

\affil[1]{Department of Mechanical and Materials Engineering, Queen's University, 45 Union St, Kingston, ON K7L 3N6}

\affil[2]{contact@richardzjm.com}
\affil[3]{17zjm1@queensu.ca}

\begin{abstract}
This template can be used to approximate final page count for submission to Optica Publishing Group’s journals \emph{Applied Optics}, \emph{Journal of Optical Communications and Networking}, JOSA A, JOSA B, \emph{Optics Letters}, \emph{Optica}, and \emph{Photonics Research}. Use the shortarticle/true option for \emph{Optics Letters} and short \emph{Optica} articles.  Authors may also \href{https://opticaopen.org}{submit articles} prepared using this template to the Optica Publishing Group preprint server, \href{https://preprints.opticaopen.org}{Optica Open}. However, doing so is optional. Please refer to the submission guidelines found there. Note that copyright and licensing information should no longer be added to your Journal or Optica Open manuscript.
\end{abstract}

\setboolean{displaycopyright}{false}
\doi{\url{https://www.richardzjm.com/projects} \\ \url{https://github.com/RichardZJM/K-MTP-training}}

\begin{document}

\maketitle

\section{Introduction}
Molecular Dynamics (MD) has been an invaluable tool in fields such as computational chemistry, materials science, and pharmaceutical drug discovery, for modelling the motions of atoms and predicting the properties of compounds \cite{karplus2002molecular}. MD models atomic motions using a classical representation of atoms, representing them as particles and applying the interatomic forces to numerically solve the equations of motion.

Quantum chemistry methods provide perhaps the best descriptor of the interatomic methods. With considerations for the electrons of the system, techniques like Density Functional Theory (DFT) and Hartree-Fock methods provide a high-fidelity force calculation by approximating a solution to the many-bodied Schrödinger Equation \cite{DFT}. However, this comes with a cubic time complexity and considerable expense even for a system of moderate size.  A single timestep in a DFT MD run of 56 atoms took several hours on 48 cores. 

Interatomic potentials, less expensive descriptors which approximate interatomic forces predictions, are usually used. In the simplest case, pair potentials such as the Lennard-Jones Potential or the Morse Potential can be used to describe the forces between all pairs of atoms with a simple function of pair-wise separation. More sophisticated many-body potentials like EAM potentials for metals and the bond order potentials for covalent solids have seen widespread practical use \cite{MatSci}. Much of recent research in interatomic potential development has been focused on machine learning (ML) potentials. These offer an improvable and adaptable approach to force modelling which is more made accessible by the relative ease of training data acquisition through quantum modelling methods. Some of the common categories of ML potentials include neural networks, Gaussian approximation potentials, and linear regression on basis sets \cite{mlip}.

The Moment Tensor Potential (MTP) falls into this last category, making considerations for the radial and angular interactions. These contributions are evaluated within a cutoff radius, taking the sum of learnable linear combinations on the atom-atom positions \cite{mtp}. The MTP architecture can be enhanced with an active learning scheme, applying the D-optimality criterion on the information matrix formed by the value of the basis functions for a set of configurations. Thus, an MTP can decide whether newly encountered configurations will be likely misrepresented or are worthwhile to learn from, selecting them on the fly for further training through DFT calculations.

\section{MTP Architecture}
Here we provide a brief overview of the MTP architecture. More rigorous representations can be obtained from the source paper \cite{mtp}.

In the MTP, the total energy of a configuration, $E^{MTP} (\textrm{cfg}) $, is represented by the total contributions, $V(n_i)$, of the local environment of each atom, shown in Equation \ref{eq:configEnergy}.  $V(n_i)$ is represented by the relative positions of atoms within a cutoff radius of the $i$th atom.

\begin{equation} \label{eq:configEnergy}
  E^{MTP} (\textrm{cfg}) = \sum_{i=1}^{n} V(n_i)
\end{equation}

For each atom, the contribution is a linear combination of a set of learnable weights, $\xi = \{\xi_\alpha\}$ and the basis set, shown in Equation \ref{eq:basisSet}.

\begin{equation} \label{eq:basisSet}
  V(n_i) = \sum_{\alpha} \xi_\alpha   B_\alpha (n_i)
\end{equation}

The elements of this basis set are formed from but not necessarily directly formed from moment tensor descriptors or simply \textit{moments}. A moment, $M_{\mu,\nu}$,  is characterized by a $\mu$ and a $\nu$ value, and has level, $\textrm{lev}_{max}$, which is given by \ref{eq:levelMTP}.

\begin{equation} \label{eq:levelMTP}
  \textrm{lev}M_{\mu,\nu} = 2 + 4\mu + \nu
\end{equation}

Each moment, given by Equation \ref{eq:moment}, has a radial part, $f_\mu (|r_{ij}|,z_i,z_j) $, characterized by the moment's $\mu$ value,  based on the distance between atoms, $|r_{ij}|$, and the species, $z_i$ and $z_j$. It also has an angular part, given by $r_{ij} ^{\otimes \nu}$.

\begin{equation} \label{eq:moment}
  M_{\mu,\nu} (n_i)= \sum_{j} f_\mu (|r_{ij}|,z_i,z_j) r_{ij} ^{\otimes \nu}
\end{equation}

\subsection{Radial Considerations}
The radial component, given by Equation \ref{eq:radial}, expands into a linear combination of a set of learnable weights $c = \{c^{(\beta)} _ {\mu,z_i,z_j} \}$, on a basis set of radial functions.

\begin{equation} \label{eq:radial}
  f_\mu (|r_{ij}|,z_i,z_j) = \sum ^ {N_o} _ {\beta = 1} c^{(\beta)} _ {\mu,z_i,z_j}  Q^{(\beta)}(|r_{ij}|)
\end{equation}

The radial functions, given in Equation \ref{eq:cheby}, are simply members of a polynomial basis set, usually Chebyshev polynomials of the first order, $\phi$, which are quadratically smoothed towards the cutoff radius, $R_{\textrm{cut}}$, to preserve a continuous second derivative.

\begin{equation} \label{eq:cheby}
  Q^{(\beta)}(|r_{ij}|)=  \begin{cases}
    \phi ^{(\beta)}(|r_{ij}|) (R_{\textrm{cut}} - |r_{ij}|)^2& |r_{ij}| < R_{\textrm{cut}} \\
    0 & |r_{ij}| \geq R_{\textrm{cut}} 
\end{cases}
\end{equation}

\subsection{Angular Considerations}
The angular component, $r_{ij} ^{\otimes \nu}$, represents the outer product of $\nu$ copies of the relative position vectors between atoms $i$ and $j$. This yields a tensor of dimension equal to $\nu$. These values are not scalars, although they could yield scalars on their contraction.

\subsection{Contraction of Tensors}
Since the contribution is reliant on a linear combination of the basis functions, moments cannot directly be applied to form the basis functions due to the tensors possibly having different ranks. Instead, contractions are performed on some number of moments, which can ultimately lead to a scalar value. Much like moments have levels, the level of a contraction of moments is the sum of the levels of the constituent moments.

When choosing the hyperparameters of an MTP, we define a maximum level, $\textrm{lev}_max$, and take all contractions of moment tensor descriptors which have a level less than or equal to this maximum level. Two other key hyperparameters in the MTP model are the cutoff radii (lower and upper bound) and the number of elements in the basis set.

\section{Active Learning}
The MTP is also capable of active learning, which allows the MTP to evaluate new configurations it encounters in use. When given a set of training configurations, each training configuration will produce a set of basis functions for each atom \cite{mlip}. These per-atom basis function values can be summed up to generate a per-configuration value of each basis function.

\begin{equation}
  E^{MTP}(\textrm{cfg}) = \sum_i \sum_\alpha^m \xi_\alpha B_\alpha (n_i) = \sum_\alpha^m \xi_\alpha (\sum_i  B_\alpha (n_i)) = \sum_\alpha^m \xi_\alpha b_\alpha
\end{equation}

Thus, for each training configuration, we can formulate a vector of the per-configuration value of the basis functions, $b(\textrm{cfg})$. Taking these vectors of rows, we can form a matrix with a height equal to the number of training configurations and a width equal to the number of basis functions in the chosen level of MTP.  For $n$ training configurations with $m$ basis functions we have that,

\[
\begin{bmatrix} 
    b_{1}(\textrm{cfg}_1)  & b_{1}(\textrm{cfg}_2)  & \cdots & b_{\alpha}(\textrm{cfg}_1) \\
    b_{2}(\textrm{cfg}_2)  & b_{2}(\textrm{cfg}_2)  & \cdots & b_{\alpha}(\textrm{cfg}_2) \\
    \vdots & \vdots & \ddots & \vdots \\ 
    b_{1}(\textrm{cfg}_n)       & b_{2}(\textrm{cfg}_n)  & \dots & b_{\alpha}(\textrm{cfg}_n) 
\end{bmatrix}
\]


In cases where there are more training configurations than we have basis functions, $m < n$, this matrix would be tall. Applying the D-optimality criterion, the $m$ rows which compose the $m$ by $m$ square submatrix, $A$, that has the largest determinant magnitude are taken to be the most linearly independent members of the training set. When the MTP is used in an MD simulation, it can assess whether a newly encountered configuration, $\textrm{cfg}_x$ would increase the magnitude of $A$'s determinant. MTP evaluates $c_i(\textrm{cfg}_x)$, the degree to which the determinant would be changed should the new configuration $x$ replace the $i$th member of the current submatrix $A$, using \ref{eq:actLearn}.

\begin{equation} \label{eq:actLearn}
  c(\textrm{cfg}_x)  = (c_1(\textrm{cfg}_x) \cdots c_m(\textrm{cfg}_x)) = (b_1(\textrm{cfg}_x) \cdots b_m(\textrm{cfg}_x)) A^{-1}
\end{equation}

The extrapolation grade is then defined at the maximum possible change in the magnitude of the determinant of $A$, This is equivalent to the maximum of the vector $c(\textrm{cfg}_x)$.

\begin{equation}
  \gamma(\textrm{cfg}_x) = \max c_1(\textrm{cfg}_x)
\end{equation}

Should the extrapolation grade be below 1, the newly encountered configuration is conceptually interpolating relative to the existing database. Grades above 1 correspond to interpolation or a new configuration. Interpolation would imply that adding the new configuration would allow the training set to encompass a richer (more linearly independent) representation, providing an evaluation criterion for determining future training on the fly. When a new configuration is added, it replaces the corresponding member in the submatrix $A$.

\section{Objectives}
As an interatomic potential of interest to the Queen's Nuclear Materials Group, we look to explore the behaviour and validity of the MTP for Potassium metal. Precisely we aim to:
\begin{itemize}
  \itemsep0em
  \item Train general-purpose MTPs of potassium metal for temperature ranges which span the solid and liquid phases, under reasonable levels of pressure and strain. 
  \item Explore the effects of various MD learning simulations and training schemes on the active learning of an MTP.
  \item  Test the sensitivity of the MTP to hyperparameters and the initial, randomized learnable parameters. 
  \item Validate the MTP with physical property prediction against DFT results: elastic properties, radial distribution function (RDF), and configuration errors.
\end{itemize}

\section{Methodology}
To produce generally applicable MTPs, and examine the practical applicability of the MTP, we emphasize the importance of a \textit{representative} training scheme for the target phases and environments. To examine potassium metal in solid and liquid phases, we target representative performance within 100-1000K, with minor strains and pressures.

We first set a baseline of 

\section{Corresponding author}

We require manuscripts to identify a single corresponding author. The corresponding author typically is the person who submits the manuscript and handles correspondence throughout the peer review and publication process. If other statements about author contribution and contact are needed, they can be added in addition to the corresponding author designation.

%Example with the corresponding author designated by an asterisk:

%\author{Author One\authormark{1} and Author Two\authormark{2,*}}

%\address{\authormark{1}Peer Review, Publications Department,
%Optica Publishing Group, 2010 Massachusetts Avenue NW,
%Washington, DC 20036, USA\\
%\authormark{2}Publications Department, Optica Publishing Group,
%2010 Massachusetts Avenue NW, Washington, DC 20036, USA\\
%%\authormark{3}xyz@optica.org}

%\email{\authormark{*}xyz@optica.org}}

%Example with the corresponding author designated by an asterisk and a note indicating equal contributions by two authors.

%\author{Author One\authormark{1,3} and Author %Two\authormark{2,3,*}}

%\address{\authormark{1}Peer Review, Publications Department,
%Optica Publishing Group, 2010 Massachusetts Avenue NW, %Washington, DC 20036, USA\\
%\authormark{2}Publications Department, Optica Publishing Group, %2010 Massachusetts Avenue NW, Washington, DC 20036, USA\\
%\authormark{3}The authors contributed equally to this work.\\
%\authormark{*}xyz@optica.org}}

%\section{Examples of Article Components}
%\label{sec:examples}

The sections below show examples of different article components.

\section{Figures and Tables}

Do not place figures and tables at the back of the manuscript. Figures and tables should be placed and sized as they are likely to appear in the final article. 

Figures and Tables should be labelled and referenced in the standard way using the \verb|\label{}| and \verb|\ref{}| commands.

\subsection{Sample Figure}

Figure \ref{fig:false-color} shows an example figure.

\begin{figure}[ht]
\centering
\fbox{\includegraphics[width=\linewidth]{opticafig1}}
\caption{Dark-field image of a point absorber.}
\label{fig:false-color}
\end{figure}

\subsection{Sample Table}

Table \ref{tab:shape-functions} shows an example table.

\begin{table}[htbp]
\centering
\caption{\bf Shape Functions for Quadratic Line Elements}
\begin{tabular}{ccc}
\hline
local node & $\{N\}_m$ & $\{\Phi_i\}_m$ $(i=x,y,z)$ \\
\hline
$m = 1$ & $L_1(2L_1-1)$ & $\Phi_{i1}$ \\
$m = 2$ & $L_2(2L_2-1)$ & $\Phi_{i2}$ \\
$m = 3$ & $L_3=4L_1L_2$ & $\Phi_{i3}$ \\
\hline
\end{tabular}
  \label{tab:shape-functions}
\end{table}

\section{Sample Equation}

Let $X_1, X_2, \ldots, X_n$ be a sequence of independent and identically distributed random variables with $\text{E}[X_i] = \mu$ and $\text{Var}[X_i] = \sigma^2 < \infty$, and let
\begin{equation}
S_n = \frac{X_1 + X_2 + \cdots + X_n}{n}
      = \frac{1}{n}\sum_{i}^{n} X_i
\label{eq:refname1}
\end{equation}
denote their mean. Then as $n$ approaches infinity, the random variables $\sqrt{n}(S_n - \mu)$ converge in distribution to a normal $\mathcal{N}(0, \sigma^2)$.

\section{Sample Algorithm}

Algorithms can be included using the commands as shown in algorithm \ref{alg:euclid}.

\begin{algorithm}
\caption{Euclid’s algorithm}\label{alg:euclid}
\begin{algorithmic}[1]
\Procedure{Euclid}{$a,b$}\Comment{The g.c.d. of a and b}
\State $r\gets a\bmod b$
\While{$r\not=0$}\Comment{We have the answer if r is 0}
\State $a\gets b$
\State $b\gets r$
\State $r\gets a\bmod b$
\EndWhile\label{euclidendwhile}
\State \textbf{return} $b$\Comment{The gcd is b}
\EndProcedure
\end{algorithmic}
\end{algorithm}

\subsection{Supplementary materials in Optica Publishing Group journals}
Optica Publishing Group journals allow authors to include supplementary materials as integral parts of a manuscript. Such materials are subject to peer-review procedures along with the rest of the paper and should be uploaded and described using the Prism manuscript system. Please refer to the \href{https://opg.optica.org/submit/style/supplementary_materials.cfm}{Author Guidelines for Supplementary Materials in Optica Publishing Group Journals} for more detailed instructions on labeling supplementary materials and your manuscript. For preprints submitted to Optica Open a link to supplemental material should be included in the submission, but do not upload the material.

\textbf{Authors may also include Supplemental Documents} (PDF documents with expanded descriptions or methods) with the primary manuscript. At this time, supplemental PDF files are not accepted for JOCN or PRJ. To reference the supplementary document, the statement ``See Supplement 1 for supporting content.'' should appear at the bottom of the manuscript (above the References heading). Supplemental documents are not accepted for Optica Open preprints.

\begin{figure}[ht!]
\centering\includegraphics{opticafig2}
\caption{Terahertz focusing metalens.}
\end{figure}


\subsection{Sample Dataset Citation}

1. M. Partridge, "Spectra evolution during coating," figshare (2014), http://dx.doi.org/10.6084/m9.figshare.1004612.

\subsection{Sample Code Citation}

2. C. Rivers, "Epipy: Python tools for epidemiology," Figshare (2014) [retrieved 13 May 2015], http://dx.doi.org/10.6084/m9.figshare.1005064.

\section{Backmatter}
Backmatter sections should be listed in the order Funding/Acknowledgment/Disclosures/Data Availability Statement/Supplemental Document section. An example of backmatter with each of these sections included is shown below.

\begin{backmatter}
\bmsection{Funding} Content in the funding section will be generated entirely from details submitted to Prism. Authors may add placeholder text in the manuscript to assess length, but any text added to this section in the manuscript will be replaced during production and will display official funder names along with any grant numbers provided. If additional details about a funder are required, they may be added to the Acknowledgments, even if this duplicates information in the funding section. See the example below in Acknowledgements. For preprint submissions, please include funder names and grant numbers in the manuscript.

\bmsection{Acknowledgments} The section title should not follow the numbering scheme of the body of the paper. Additional information crediting individuals who contributed to the work being reported, clarifying who received funding from a particular source, or other information that does not fit the criteria for the funding block may also be included; for example, ``K. Flockhart thanks the National Science Foundation for help identifying collaborators for this work.''

\bmsection{Disclosures} Disclosures should be listed in a separate section at the end of the manuscript. List the Disclosures codes identified on the \href{https://opg.optica.org/submit/review/conflicts-interest-policy.cfm}{Conflict of Interest policy page}. If there are no disclosures, then list ``The authors declare no conflicts of interest.''

\smallskip

\noindent Here are examples of disclosures:


\bmsection{Disclosures} ABC: 123 Corporation (I,E,P), DEF: 456 Corporation (R,S). GHI: 789 Corporation (C).

\bmsection{Disclosures} The authors declare no conflicts of interest.


\bmsection{Data Availability Statement} A Data Availability Statement (DAS) will be required for all submissions beginning 1 March 2021. The DAS should be an unnumbered separate section titled ``Data Availability'' that
immediately follows the Disclosures section. See the \href{https://opg.optica.org/submit/review/data-availability-policy.cfm}{Data Availability Statement policy page} for more information.

There are four common (sometimes overlapping) situations that authors should use as guidance. These are provided as minimal models, and authors should feel free to
include any additional details that may be relevant.



\begin{enumerate}
\item When datasets are included as integral supplementary material in the paper, they must be declared (e.g., as "Dataset 1" following our current supplementary materials policy) and cited in the DAS, and should appear in the references.

\bmsection{Data availability} Data underlying the results presented in this paper are available in Dataset 1, Ref. [3].

\item When datasets are cited but not submitted as integral supplementary material, they must be cited in the DAS and should appear in the references.

\bmsection{Data availability} Data underlying the results presented in this paper are available in Ref. [3].

\item If the data generated or analyzed as part of the research are not publicly available, that should be stated. Authors are encouraged to explain why (e.g.~the data may be restricted for privacy reasons), and how the data might be obtained or accessed in the future.

\bmsection{Data availability} Data underlying the results presented in this paper are not publicly available at this time but may be obtained from the authors upon reasonable request.

\item If no data were generated or analyzed in the presented research, that should be stated.

\bmsection{Data availability} No data were generated or analyzed in the presented research.
\end{enumerate}

\bigskip

\noindent Data availability statements are not required for preprint submissions.

\bmsection{Supplemental document}
See Supplement 1 for supporting content. 

\end{backmatter}

\section{References}

Note that \emph{Optics Letters} and \emph{Optica} short articles use an abbreviated reference style. Citations to journal articles should omit the article title and final page number; this abbreviated reference style is produced automatically when the \emph{Optics Letters} journal option is selected in the template, if you are using a .bib file for your references.

However, full references (to aid the editor and reviewers) must be included as well on a fifth informational page that will not count against page length; again this will be produced automatically if you are using a .bib file.

\bigskip
\noindent Add citations manually or use BibTeX. See \cite{Zhang:14,OPTICA,FORSTER2007,testthesis,manga_rao_single_2007}.

% Bibliography
\bibliography{sample}

% Full bibliography added automatically for Optics Letters submissions; the following line will simply be ignored if submitting to other journals.
% Note that this extra page will not count against page length
\bibliographyfullrefs{sample}

%Manual citation list
%\begin{thebibliography}{1}
%\bibitem{Zhang:14}
%Y.~Zhang, S.~Qiao, L.~Sun, Q.~W. Shi, W.~Huang, %L.~Li, and Z.~Yang,
 % \enquote{Photoinduced active terahertz metamaterials with nanostructured
  %vanadium dioxide film deposited by sol-gel method,} Opt. Express \textbf{22},
  %11070--11078 (2014).
%\end{thebibliography}

% Please include bios and photos of all authors for aop articles
\ifthenelse{\equal{\journalref}{aop}}{%
\section*{Author Biographies}
\begingroup
\setlength\intextsep{0pt}
\begin{minipage}[t][6.3cm][t]{1.0\textwidth} % Adjust height [6.3cm] as required for separation of bio photos.
  \begin{wrapfigure}{L}{0.25\textwidth}
    \includegraphics[width=0.25\textwidth]{john_smith.eps}
  \end{wrapfigure}
  \noindent
  {\bfseries John Smith} received his BSc (Mathematics) in 2000 from The University of Maryland. His research interests include lasers and optics.
\end{minipage}
\begin{minipage}{1.0\textwidth}
  \begin{wrapfigure}{L}{0.25\textwidth}
    \includegraphics[width=0.25\textwidth]{alice_smith.eps}
  \end{wrapfigure}
  \noindent
  {\bfseries Alice Smith} also received her BSc (Mathematics) in 2000 from The University of Maryland. Her research interests also include lasers and optics.
\end{minipage}
\endgroup
}{}


\end{document}
